% Options for packages loaded elsewhere
\PassOptionsToPackage{unicode}{hyperref}
\PassOptionsToPackage{hyphens}{url}
%
\documentclass[
]{article}
\usepackage{amsmath,amssymb}
\usepackage{lmodern}
\usepackage{iftex}
\ifPDFTeX
  \usepackage[T1]{fontenc}
  \usepackage[utf8]{inputenc}
  \usepackage{textcomp} % provide euro and other symbols
\else % if luatex or xetex
  \usepackage{unicode-math}
  \defaultfontfeatures{Scale=MatchLowercase}
  \defaultfontfeatures[\rmfamily]{Ligatures=TeX,Scale=1}
\fi
% Use upquote if available, for straight quotes in verbatim environments
\IfFileExists{upquote.sty}{\usepackage{upquote}}{}
\IfFileExists{microtype.sty}{% use microtype if available
  \usepackage[]{microtype}
  \UseMicrotypeSet[protrusion]{basicmath} % disable protrusion for tt fonts
}{}
\makeatletter
\@ifundefined{KOMAClassName}{% if non-KOMA class
  \IfFileExists{parskip.sty}{%
    \usepackage{parskip}
  }{% else
    \setlength{\parindent}{0pt}
    \setlength{\parskip}{6pt plus 2pt minus 1pt}}
}{% if KOMA class
  \KOMAoptions{parskip=half}}
\makeatother
\usepackage{xcolor}
\usepackage[margin=1in]{geometry}
\usepackage{longtable,booktabs,array}
\usepackage{calc} % for calculating minipage widths
% Correct order of tables after \paragraph or \subparagraph
\usepackage{etoolbox}
\makeatletter
\patchcmd\longtable{\par}{\if@noskipsec\mbox{}\fi\par}{}{}
\makeatother
% Allow footnotes in longtable head/foot
\IfFileExists{footnotehyper.sty}{\usepackage{footnotehyper}}{\usepackage{footnote}}
\makesavenoteenv{longtable}
\usepackage{graphicx}
\makeatletter
\def\maxwidth{\ifdim\Gin@nat@width>\linewidth\linewidth\else\Gin@nat@width\fi}
\def\maxheight{\ifdim\Gin@nat@height>\textheight\textheight\else\Gin@nat@height\fi}
\makeatother
% Scale images if necessary, so that they will not overflow the page
% margins by default, and it is still possible to overwrite the defaults
% using explicit options in \includegraphics[width, height, ...]{}
\setkeys{Gin}{width=\maxwidth,height=\maxheight,keepaspectratio}
% Set default figure placement to htbp
\makeatletter
\def\fps@figure{htbp}
\makeatother
\setlength{\emergencystretch}{3em} % prevent overfull lines
\providecommand{\tightlist}{%
  \setlength{\itemsep}{0pt}\setlength{\parskip}{0pt}}
\setcounter{secnumdepth}{-\maxdimen} % remove section numbering
\ifLuaTeX
  \usepackage{selnolig}  % disable illegal ligatures
\fi
\IfFileExists{bookmark.sty}{\usepackage{bookmark}}{\usepackage{hyperref}}
\IfFileExists{xurl.sty}{\usepackage{xurl}}{} % add URL line breaks if available
\urlstyle{same} % disable monospaced font for URLs
\hypersetup{
  pdftitle={CodeBook},
  pdfauthor={Alexander Cormack},
  hidelinks,
  pdfcreator={LaTeX via pandoc}}

\title{CodeBook}
\author{Alexander Cormack}
\date{2022-11-01}

\begin{document}
\maketitle

\hypertarget{the-source-data}{%
\section{The Source Data}\label{the-source-data}}

The source data used for this assignment consists essentially of seven
files

\hypertarget{features.txt}{%
\subsubsection{features.txt}\label{features.txt}}

The file contains a dataframe with 561 observations and 2 variables. V1
is an integer variable containing the integers from 1 to 561 in
ascending order V2 is a character vector listing all of the features
measured in the Samsung Galaxy S smartphone accelerometer trial. The
``features.info.txt'' file contains the following information about the
features measured.

\hypertarget{feature-selection}{%
\section{Feature Selection}\label{feature-selection}}

The features selected for this database come from the accelerometer and
gyroscope 3-axial raw signals tAcc-XYZ and tGyro-XYZ. These time domain
signals (prefix `t' to denote time) were captured at a constant rate of
50 Hz. Then they were filtered using a median filter and a 3rd order low
pass Butterworth filter with a corner frequency of 20 Hz to remove
noise. Similarly, the acceleration signal was then separated into body
and gravity acceleration signals (tBodyAcc-XYZ and tGravityAcc-XYZ)
using another low pass Butterworth filter with a corner frequency of 0.3
Hz.

Subsequently, the body linear acceleration and angular velocity were
derived in time to obtain Jerk signals (tBodyAccJerk-XYZ and
tBodyGyroJerk-XYZ). Also the magnitude of these three-dimensional
signals were calculated using the Euclidean norm (tBodyAccMag,
tGravityAccMag, tBodyAccJerkMag, tBodyGyroMag, tBodyGyroJerkMag).

Finally a Fast Fourier Transform (FFT) was applied to some of these
signals producing fBodyAcc-XYZ, fBodyAccJerk-XYZ, fBodyGyro-XYZ,
fBodyAccJerkMag, fBodyGyroMag, fBodyGyroJerkMag. (Note the `f' to
indicate frequency domain signals).

These signals were used to estimate variables of the feature vector for
each pattern:\\
`-XYZ' is used to denote 3-axial signals in the X, Y and Z directions.

tBodyAcc-XYZ tGravityAcc-XYZ tBodyAccJerk-XYZ tBodyGyro-XYZ
tBodyGyroJerk-XYZ tBodyAccMag tGravityAccMag tBodyAccJerkMag
tBodyGyroMag tBodyGyroJerkMag fBodyAcc-XYZ fBodyAccJerk-XYZ
fBodyGyro-XYZ fBodyAccMag fBodyAccJerkMag fBodyGyroMag fBodyGyroJerkMag

The set of variables that were estimated from these signals are:

mean(): Mean value std(): Standard deviation mad(): Median absolute
deviation max(): Largest value in array min(): Smallest value in array
sma(): Signal magnitude area energy(): Energy measure. Sum of the
squares divided by the number of values. iqr(): Interquartile range
entropy(): Signal entropy arCoeff(): Autorregresion coefficients with
Burg order equal to 4 correlation(): correlation coefficient between two
signals maxInds(): index of the frequency component with largest
magnitude meanFreq(): Weighted average of the frequency components to
obtain a mean frequency skewness(): skewness of the frequency domain
signal kurtosis(): kurtosis of the frequency domain signal
bandsEnergy(): Energy of a frequency interval within the 64 bins of the
FFT of each window. angle(): Angle between to vectors.

Additional vectors obtained by averaging the signals in a signal window
sample. These are used on the angle() variable:

gravityMean tBodyAccMean tBodyAccJerkMean tBodyGyroMean
tBodyGyroJerkMean

The complete list of variables of each feature vector is available in
`features.txt'

\hypertarget{subjects_test.txt}{%
\subsubsection{subjects\_test.txt}\label{subjects_test.txt}}

The file contains a dataframe with 2947 observations and 1 variable. V1
is an integer variable representing the subjects for which measurements
were made in the test pool. The file contains the following unique
values:

\begin{longtable}[]{@{}ll@{}}
\toprule()
V1 & n \\
\midrule()
\endhead
2 & 302 \\
4 & 317 \\
9 & 288 \\
10 & 294 \\
12 & 320 \\
13 & 327 \\
18 & 364 \\
20 & 354 \\
24 & 381 \\
\bottomrule()
\end{longtable}

\hypertarget{subjects_train.txt}{%
\subsubsection{subjects\_train.txt}\label{subjects_train.txt}}

The file contains a dataframe with 7352 observations and 1 variable. V1
is an integer variable representing the subjects for which measurements
were made in the train pool. The file contains the following unique
values:

\begin{longtable}[]{@{}ll@{}}
\toprule()
V1 & n \\
\midrule()
\endhead
1 & 347 \\
3 & 341 \\
4 & 302 \\
6 & 325 \\
7 & 308 \\
8 & 281 \\
11 & 316 \\
14 & 323 \\
15 & 328 \\
16 & 366 \\
17 & 368 \\
19 & 360 \\
21 & 408 \\
22 & 321 \\
23 & 372 \\
25 & 409 \\
26 & 392 \\
27 & 376 \\
28 & 382 \\
29 & 344 \\
30 & 383 \\
\bottomrule()
\end{longtable}

\hypertarget{y_test.txt}{%
\subsubsection{y\_test.txt}\label{y_test.txt}}

The file contains a dataframe with 2947 observations and 1 variable. V1
is an integer variable (1-6) representing the activities for which
measurements were made in the test pool.

The ``activity\_labels.txt'' maps the integer values to the specfic
activities:

1 WALKING 2 WALKING\_UPSTAIRS 3 WALKING\_DOWNSTAIRS 4 SITTING 5 STANDING
6 LAYING

The file contains the following unique values:

\begin{longtable}[]{@{}ll@{}}
\toprule()
V1 & n \\
\midrule()
\endhead
1 & 496 \\
2 & 471 \\
3 & 420 \\
4 & 491 \\
5 & 532 \\
6 & 537 \\
\bottomrule()
\end{longtable}

\hypertarget{y_train.txt}{%
\subsubsection{y\_train.txt}\label{y_train.txt}}

The file contains a dataframe with 7352 observations and 1 variable. V1
is an integer variable (1-6) representing the activities for which
measurements were made in the train pool.

The ``activity\_labels.txt'' maps the integer values to the specfic
activities:

1 WALKING 2 WALKING\_UPSTAIRS 3 WALKING\_DOWNSTAIRS 4 SITTING 5 STANDING
6 LAYING

The file contains the following unique values:

\begin{longtable}[]{@{}ll@{}}
\toprule()
V1 & n \\
\midrule()
\endhead
1 & 1226 \\
2 & 1073 \\
3 & 986 \\
4 & 1286 \\
5 & 1374 \\
6 & 1407 \\
\bottomrule()
\end{longtable}

\hypertarget{x_test.txt}{%
\subsubsection{X\_test.txt}\label{x_test.txt}}

The file contains a dataframe with 2947 observations and 561 variables.
The variables are all numeric floats ranging from 1 to -1 each with 9
decimal places. The variables map to the ``features.txt file'' above.
The observations map to the ``subject.txt'' and ``activity.txt'' files
above.

\hypertarget{x_train.txt}{%
\subsubsection{X\_train.txt}\label{x_train.txt}}

The file contains a dataframe with 2947 observations and 561 variables.
The variables are all numeric floats ranging from 1 to -1 each with 9
decimal places. The variables map to the ``features.txt file'' above.
The observations map to the ``subject.txt'' and ``activity.txt'' files
above.

\end{document}
